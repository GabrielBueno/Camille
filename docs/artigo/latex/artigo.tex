\documentclass[12pt, a4paper, titlepage]{article}

\usepackage[brazilian]{babel}
\usepackage[utf8]{inputenc}
\usepackage[T1]{fontenc}
\usepackage{url}
\usepackage[backend=biber,style=abnt]{biblatex}
\usepackage{csquotes}
\usepackage{authblk}
\usepackage[a4paper, left=3cm, right=2cm, top=3cm, bottom=2cm]{geometry}
\usepackage{setspace}
\usepackage{multirow}
\usepackage{listings}

\onehalfspacing

\addbibresource{artigo.bib}

\author{Gabriel Almeida Bueno}
\affil{FATEC Zona Sul}
\title{Tecnologia e Artes, um estudo sobre a tecnologia da informação como meio para compreensão e realização artística}

\begin{document}

\maketitle

\section{Introdução}
% O que é arte? Qual é a sua importância?
% Conceitos de arte para os antigos
% Arte e as 12 inteligências
% Tecnologia inerente À arte
% Tecnologia como apoio À didática
% Tecnologia como meio de realização artística
% Arte computacional
% IA (NNs e GANs)

\section{Referencial teórico}

\nocite{*}
\printbibliography

\end{document}