\documentclass[12pt, a4paper, titlepage]{article}

\usepackage[brazilian]{babel}
\usepackage[utf8]{inputenc}
\usepackage[T1]{fontenc}
\usepackage{url}
\usepackage[backend=biber,style=abnt]{biblatex}
\usepackage{csquotes}
\usepackage{authblk}
\usepackage[a4paper, left=3cm, right=2cm, top=3cm, bottom=2cm]{geometry}
\usepackage{setspace}
\usepackage{multirow}
\usepackage{listings}

\onehalfspacing

\addbibresource{artigo.bib}

\author{Gabriel Almeida Bueno}
\affil{FATEC Zona Sul}
\title{Tecnologia e Artes, um estudo sobre a tecnologia da informação como meio para compreensão e realização artística}

\begin{document}

\maketitle

\section{Introdução}
% O que é arte? Qual é a sua importância?
A arte é parte indissociável da vivência humana, e a tecnologia é parte indissociável da arte. 
Sendo uma das atividades mais antigas exercidas pelo ser-humano, podemos enxergar características estéticas e manifestações artísticas realizadas pelos vários povos e culturas antigas até a contemporaneidade, seja por meio do artesanato, arquitetura, pintura ou poesia.
A harmonia e simetria foram sempre benquistos por qualquer um que seja,
independente do seu meio social ou gostos individuais, havendo manifestações estéticas em todo o grupo social, sejam nas culturas populares ou nas ditas belas artes.
Na Poética, ao definir a arte da poesia, Aristóteles afirma que 
\emph{"as coisas que observamos ao natural e nos fazem pena agradam-nos quando as vemos representadas em imagens muito perfeitas"} \cite[p.42]{aristotle_poetics}.
Cada registro artístico, porém, representa não só algo que é sensivelmente belo, mas constitui uma expressão do indivíduo que a fez e da época
e meio em que estava inserido. A arte mostra-se, portanto, de valor inestimável como registro da expressão humana:
\emph{"Os frutos da pintura podem ser compreendidos por todas as populações do universo pois seus resultados
são sujeitos ao poder da visão [...] não necessitando de intérpretes para as várias línguas"} \cite{davinci_thoughtsonart}. 
Identidades religiosas e nacionais também fazem uso da estética, já que historicamente podemos observar que
\emph{"é nos trabalhos de arte que nações tem depositado as mais ricas intuições e ideias que possuem; e não 
infrequentemente as belas artes fornecem uma chave para a interpretação da sabedoria e religião dos povos"} \cite{hegel}.
O ato de realizar arte, porém, é estritamente ligado à tecnologia.
As ferramentas criadas pelo homem a fim de subjugar os obstáculos impostos
pelo meio ambiente à sua sobrevivência, foram e sempre serão usadas pelo artista como meio de expressão e para o fazer artístico \cite{gouzouasis}.
Podemos notar isso observando os registros de arte ao longo da história. Das pinturas que passaram das paredes das cavernas para o óleo em tela, até
a fotografia; da música tocada em alaúdes com tripas torcidas para os violões com cordas de nylon, até as guitarras elétricas; 
da gravação e reprodução sonora que partiu do fonógrafo para os computadores e CDs, até o \emph{streaming}. Podemos observar como a tecnologia 
de uma época pode influenciar nas manifestações artísticas do período. Um dos sentidos que o famoso aforismo de McLuhan, "o meio é a mensagem", 
carrega em si é o de que o \emph{"meio transforma o seu conteúdo"} \cite[p.50]{braga_mcluhan}. Portanto, podemos notar o quanto um novo meio, fruto de
uma inovação tecnológica, influencia na própria mensagem passada na obra artística.
Estamos na era da informação, com capacidade computacional de sobra e 
uma digitalização crescente do mundo tangível, como a tecnologia contemporânea pode influenciar no estado atual da realização e compreensão artística?
Como a arte deve passar a ser abordada na educação artística e desenvolvimento do senso estético e crítico do indivíduo em uma realidade onde o digital
é cada vez mais a norma?


Acreditamos que há uma crescente importância na abordagem da tecnologia na sala de aula, já que a sociedade ruma à normatização tecnológica.
A tecnologia pode servir tanto de auxílio como forma de realização de arte.

Pirâmide de Barbosa

Até mesmo o mais romântico cientista pode enxergar alguma forma de arte na atividade que exerce.

\emph{"Um matemático, como um pintor ou poeta, é um criador de padrões. Se os padrões daquele são mais permanentes do que os destes, é porque eles são feitos com ideias"}
\cite{hardy_apology}

Lockhart lamenta a situação do ensino da matemática (sua lamentação pode muito bem ser transposta para o próprio ensino da arte):
\emph{"Nenhuma sociedade jamais reduziria uma forma tão bela e significativa de arte para algo tão insignificante e trivial. Nenhuma cultura poderia ser tão cruel
com suas crianças a ponto de privá-las de um meio tão satisfatório e natural de expressão humana. Que absurdo!"}
\cite{lockhart_lament}

% Importância da tecnologia + arte
% Arteeducação no Brasil

% Hipótese: é necessário abordar tecnologia como forma artística nas aulas
% Proposta: criar uma IA que prova que é capaz de integrar essas duas áreas

% Tecnologia como meio de realização artística (arte computacional)
% Arte computacional


\section{Referencial teórico}
% IA (NNs e GANs)


% \nocite{*}
\printbibliography

\end{document}