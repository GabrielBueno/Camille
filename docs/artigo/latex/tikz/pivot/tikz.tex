\documentclass{article}

\usepackage[english, brazilian]{babel}
\usepackage[utf8]{inputenc}
\usepackage[T1]{fontenc}
\usepackage{url}
\usepackage[backend=biber,style=abnt]{biblatex}
\usepackage[autostyle]{csquotes}
\usepackage{authblk}
\usepackage[a4paper, left=3cm, right=2cm, top=3cm, bottom=2cm]{geometry}
\usepackage{setspace}
\usepackage{multirow}
\usepackage{listings}
\usepackage{graphicx}
\usepackage{subcaption}
\usepackage{caption}
\usepackage{wrapfig}
\usepackage{mathtools}
\usepackage{pgfplots}
\usepackage{pgf-pie}
\usepackage{float}
\usepackage{blindtext}
\usepackage{enumitem}
\usepackage{scrextend}
\usepackage{tasks}
\usepackage{titling}
\usepackage{indentfirst}

\pagestyle{empty}
\begin{document}
	\begin{figure}[H]
		\centering
		\begin{tikzpicture}
			\begin{axis}[xlabel=$x$, ylabel=$p(x)$]
				\addplot[] {x > 0 ? 1 : 0};
			\end{axis}
		\end{tikzpicture}
		\caption{Função de ativação do perceptron de Rosenblatt.}
	\end{figure}

	\newpage
	\begin{figure}[H]
		\centering
		\begin{tikzpicture}
			\begin{axis}[xlabel=$x$, ylabel=$\sigma(x)$]
				\addplot[] {1/(1+exp(-x))};
			\end{axis}
		\end{tikzpicture}
		\caption{Função sigmoide.}
	\end{figure}

	\newpage
	\begin{figure}[H]
		\centering
		\begin{tikzpicture}
			\begin{axis}[xlabel=$x$, ylabel=$ReLU(x)$]
				\addplot[] {x > 0 ? x : 0};
			\end{axis}
		\end{tikzpicture}
		\caption{Função ReLU.}
		\end{figure}

	\newpage
	\begin{figure}[H]
		\centering
		\begin{tikzpicture}
			\pie{
				7.9/40 a 49,
				27/18 a 21,
				31.7/30 a 39,
				33.3/22 a 29
			}
		\end{tikzpicture}
		\caption{Idade dos entrevistados.}
		\label{fig:q1}
	\end{figure}

	\newpage
	\begin{figure}[H]
		\centering
		\begin{tikzpicture}
			\pie{
				1.6/Prefiro não dizer,
				1.6/Não-binário,
				20.6/Feminino,
				76.2/Masculino
			}
		\end{tikzpicture}
		\caption{Gênero dos entrevistados.}
		\label{fig:q2}
	\end{figure}

	\newpage
	\begin{figure}[H]
		\centering
		\begin{tikzpicture}
			\pie{
				6.3/Médio completo,
				46/Superior incompleto,
				47.6/Superior completo
			}
		\end{tikzpicture}
		\caption{Escolaridade dos entrevistados.}
		\label{fig:q3}
	\end{figure}

	\newpage
	\begin{figure}[H]
		\centering
		\begin{tikzpicture}
			\pie{
				1.6/Não cursaram,
				1.6/Não lembram,
				96.8/Cursaram
			}
		\end{tikzpicture}
		\caption{Distribuição da oferta de disciplinas de educação artística no ensino fundamental entre os entrevistados.}
		\label{fig:q4}
	\end{figure}

	\newpage
	\begin{figure}[H]
		\centering
		\begin{tikzpicture}
			\pie{
				3.2/Não cursaram,
				39.7/Não gostavam,
				57.1/Gostavam
			}
		\end{tikzpicture}
		\caption{Sentimento dos entrevistados em relação às aulas oferecidas no ensino fundamental.}
		\label{fig:q5}
	\end{figure}

	\newpage
	\begin{figure}[H]
		\centering
		\begin{tikzpicture}
			\pie{
				3.2/Não lembram,
				19/Não cursaram,
				77.8/Cursaram
			}
		\end{tikzpicture}
		\caption{Distribuição da oferta de disciplinas de educação artística no ensino médio entre os entrevistados.}
		\label{fig:q6}
	\end{figure}

	\newpage
	\begin{figure}[H]
		\centering
		\begin{tikzpicture}
			\pie{
				19/Não cursaram,
				28.6/Não gostavam,
				52.4/Gostavam
			}
		\end{tikzpicture}
		\caption{Sentimento dos entrevistados em relação às aulas oferecidas no ensino médio.}
		\label{fig:q7}
	\end{figure}

	\newpage
	\begin{figure}[H]
		\centering
		\begin{tikzpicture}
			\pie{
				3.2/Nem um pouco,
				30.2/Um pouco,
				66.7/Essencial
			}
		\end{tikzpicture}
		\caption{Importância atribuída pelos entrevistados à educação artística na formação de um indivíduo.}
		\label{fig:q8}
	\end{figure}

	\newpage
	\begin{figure}[H]
		\centering
		\begin{tikzpicture}
			\pie{
				3.2/Não traria benefício,
				96.8/Traria benefício
			}
		\end{tikzpicture}
		\caption{Opinião dos entrevistados sobre o benefício da integração da tecnologia no modelo convencional de educação.}
		\label{fig:q9}
	\end{figure}

	\newpage
	\begin{figure}[H]
		\centering
		\begin{tikzpicture}
			\pie{
				7.9/Traria interesse,
				92.1/Não traria interesse
			}
		\end{tikzpicture}
		\caption{Opinião dos entrevistados sobre se a integração da tecnologia com a educação artística fariam com que os alunos se interessasem mais nos assuntos abordados.}
		\label{fig:q10}
	\end{figure}
\end{document}